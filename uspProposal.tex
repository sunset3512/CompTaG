% Kyla Andersen USP proposal
\documentclass[12pt]{article}
\usepackage[margin=1in]{geometry}
\usepackage{setspace}
\usepackage{amsmath, amssymb}
\usepackage{graphicx}
\usepackage{hyperref}
\usepackage{natbib}
\usepackage{wrapfig}
\bibliographystyle{plain}

% Begin document
\begin{document}

% Title Section
\title{Topological Trait Estimation for Phylogenetic Reconstruction of Fossil
Evolution}
\author{Kyla Andersen}
\date{08/30/2024}
\maketitle

% Double spacing
\doublespacing

% Section 1: Introduction
\section{Introduction}
\begin{wrapfigure}{R}{0.35\textwidth}
	\centering
	\vspace{-12pt}
	\includegraphics[width=0.35\textwidth]{newDinoBone.png}
	\caption{A  3D scan of the left maxilla of an Edmontosaurus \citep{organ2024}.}
	\label{fig:dino_maxilla}
\end{wrapfigure}
\vspace{-15pt}

Topological methods have emerged as powerful tools in data analysis,
particularly for complex, high-dimensional datasets such as those found across
evolutionary biology. The study of phylogenetics, traditionally reliant on
morphological analysis of fossilized bones, is an area ripe for innovation
using bones from extinct species. \autoref{fig:dino_maxilla} shows how a 
dinosaur fossil can be scanned and turned into a 3D mesh. Estimating 
morphological traits from 3D scans of bones can provide insights into the 
evolutionary relationships among different species and offer a clearer picture 
of their lineage and adaptations over time, which is essential for understanding
evolution and biodiversity. This is often visualized using a phylogenetic tree, 
a diagram that illustrates the evolutionary paths and connections among species.
These diagrams are universal to all fields of biology, from virology and cancer
biology, to genomics and ecology.

Traditional approaches to this problem involve manually entering and comparing the 
morphological data, a process that is not only time-consuming but also prone to
human error and bias \citep{asher2022,bates2009}. Topological data analysis (TDA), 
offers a solution through a set  of mathematical tools that capture the intrinsic 
geometric and topological properties of shapes, allowing for an efficient method 
of trait estimation. Unlike traditional methods, TDA can handle variations in shape, 
scale, and orientation, which enables more accurate comparisons of complex 
structures \citep{zomorodian2009}. The Euler Characteristic Transform (ECT), 
a specific tool in TDA, is particularly relevant for summarizing shape information 
and distinguishing between different shapes \citep{cisewski2023}.

In analyzing the traits of fossils, TDA would allow researchers to quantify
differences and similarities by representing the shapes as toplogical spaces. By 
treating these shapes as topological spaces, it would become possible to construct a 
phylogenetic tree that reflects the evolutionary history of a given species 
with greater precision and less manual intervention \citep{yang2012}.
Specifically, I look to use the ECT as a data descriptor to explore and analyze
these topological spaces. By applying the ECT to traits found on fossils, I 
will investigate which topological properties are more informative for accurately 
mapping evolutionary branches and relationships. 

%By examining the traits found on fossils, I will 
%investigate which topological properties are most helpful for accurately mapping
%evolutionary branches and relationships. I am particularly interested in this
%area for its potential to reveal new insights into evolutionary patterns and
%connections. 

% Section 2: Background
\section{Background}
\begin{wrapfigure}{R}{0.3\textwidth}
	\centering
	\vspace{-12pt}
	\includegraphics[width=0.3\textwidth]{arteryTree.png}
	\caption{By using tools from TDA, \cite{bendich2016} uncovered previously
unknown traits of the complex branching structure of arteries.}
	\label{fig:artery_tree}
\end{wrapfigure}
\vspace{-12pt}

Recent studies have demonstrated the power of TDA in fields such as neuroscience
and anatomy. Notably, Bendich et al. \cite{bendich2016} used
TDA to analyze brain artery structures, revealing correlations between age, sex, 
and the bending of arteries that were not present in previous studies. In
\autoref{fig:artery_tree}, we see the construction of a 3D artery model which
provides a basis for applying analytical techniques. TDA methods provide new 
insights into complex data by capturing changes within interconnected 
structures and enabling the analysis of high-dimensional characteristics, 
such as shape and form, often overlooked by traditional metrics. 

%\begin{wrapfigure}{R}{0.3\textwidth}
%	\centering
%	\includegraphics[width=0.3\textwidth]{arteryTree.png}
%	\caption{By using tools from TDA, \cite{bendich2016} uncovered previously
%unknown traits of the complex branching structure of arteries, as seen here}
%	\label{fig:artery_tree}
%\end{wrapfigure}

With advancements in 3D imaging and data acquisition technologies, applying TDA 
to 3D data has gained significant traction in recent years. A study produced by 
Ou Li \citep{li2021} introduced a method that leverages TDA for segmenting and 
simplifying 3D point cloud data, a representation commonly only obtained through 
low-cost 3D sensors. The study demonstrated that TDA can effectively partition 
complex, disorganized 3D data into subsets with consistent characteristics, 
using various techniques that simplify the data structure while retaining 
critical geometric information \citep{li2021}.

Dornbusch et al. \cite{dornbusch2007}, developed a method that utilizes 3D point 
clouds to extract and parameterize morphological traits of plant organs, which 
are then used to create a comprehensive architectural model of the plant. Fitting 
these 3D point clouds into a structural model, the method enables a precise 
portrayal of the leaf and stem shape and their spatial orientations \citep{dornbusch2007}. 
By applying this technique to 3D scans of fossils, one can extract and parameterize 
various morphological traits, such as bone structure and dimensions, with high 
accuracy. This detailed characterization allows for reconstructing more exact
evolutionary trees and a better understanding of the anatomical variations among
different species. Integrating this approach will significantly enhance the
ability to trace evolutionary relationships and track the development of
specific traits over time. 

While computational methods for analyzing 3D scans have advanced significantly
in fields like medical imaging and industrial design, paleontology has not yet
fully leveraged these tools. Current research in paleontology has a noticeable lack of
specialized algorithms and methods tailored for interpreting the complex,
irregular shapes of fossilized remains, such as dinosaur bones, though scans are
routinely used when describing specimens. The gap lies in
the adaptation and development of these advanced computational methods to suit
the specific challenges of complex data. The groundwork for inferring
trees from high dimensional continuous data is well established and has been
shown to generate well supported phylogenies \cite{gonzalez2008}.

Building on existing research in 3D scan analysis and computational methods, my
goal is to advance these techniques more specifically for the field of
evolutionary biology. By applying and adapting algorithms such as the one used by
Dornbusch et al. \citep{dornbusch2007}, I aim to develop new methods for analyzing 
and characterizing dinosaur bones from 3D scans. This work seeks to fill the 
current gap in applying these advanced techniques to paleontological data, 
ultimately improving the accuracy of evolutionary trees and our understanding of 
prehistoric life. I will ensure the accuracy of my phylogenetic tree through
comparisons of established trees.

I look forward to expanding my research experience by taking on the challenge of
an independent project for the first time. With a solid background in
coding with Python, Java, and C++, I have the skills that will be instrumental 
in developing and implementing the new algorithms necessary for this research. 
This project will allow me to deepen my knowledge of computational topology and 
enhance my skills in algorithm development and data analysis. 

% Section 3: Methods
\section{Methods}
To begin this project, I will preprocess 3D scans of dinosaur bones to clean and
normalize the data. Specifically, I will ensure consistency in scale, orientation, 
and resolution by removing noise, filling holes, and aligning scans.  Next, I 
will adapt existing algorithms, such as point cloud registration to identify key 
morphological features of the bones. This will involve implementing these
algorithms in Python, utilizing libraries like NumPy, SciPy. Following this, 
I will apply TDA techniques, focusing on the ECT. The ECT will
allow me to track how the Euler characteristic changes as a filtration parameter
varies. This will provided a detailed understanding of each bone's intrinsic
properties. Additional techniques that will be used to analyze and compare
features include principal component analysis (PCA) and singular value 
decomposition (SVD). Experiments will validate the effectiveness of these methods 
by comparing them to known data, using metrics like shape similarity scores to 
ensure robustness across different types of bones.
The data collected will then be used to generate a Bayesian Brownian motion
phylogenetic model in the scripting language RevBayes, which will produce
posterior distributions of phylogenies give the data and the model. 
The final stage will involve documenting the findings in an abstract or research 
paper, which will highlight the potential applications of these methods in 
paleontology and beyond.

% Section 4: Timeline
\section{Timeline}
An outline of the estimated timeline for this project is given below.

%\begin{enumerate}
%	\item Clean, normalize, and align 3D scans using trimesh and pygltflib: 3 weeks
%	\item Implement and adapt existing algorithms to identify key morphological
%	features of the bones: 4 weeks
%	\item Develop and customize new algorithms tailored to paleontology using
%	the ETC, PCA, and the SVD to enhance automated characterization: 7 weeks
%	\item Conduct experiments to validate methods, comparing them to known data:
%	5 weeks
%	\item Analyze the results, refine methods as needed, and compile findings
%	for documentation: 6 weeks
%	\item Assemble my finding for an abstract or research paper: 1 week
%\end{enumerate}

\begin{table}[h!]
\centering
%\normalsize
%\resizebox{\textwidth}{!}{
\setlength{\tabcolsep}{10pt}
\renewcommand{\arraystretch}{1}
\begin{tabular}{|p{12cm}|p{2.5cm}|}
\hline
\textbf{Task} & \textbf{Duration} \\
\hline
Clean, normalize, and align 3D scans using trimesh and pygltflib & 3 weeks \\
Implement and adapt new existing algorithms to identify key morphological
features & 4 weeks \\
Develop and customize new algorithms tailored to paleontology using the ECT,
PCA, and the SVD to enhance automated characterization & 7 weeks \\
Conduct experiments to validate methods, comparing them to known data & 5 weeks \\
Analyze the results, refine methods as needed, and compile findings for
documentation & 6 weeks \\
Assemble my findings for an abstract or research paper & 2 weeks \\
\hline
\end{tabular}
%}
\end{table}

% Section 5: Collaboration with Faculty Sponsor
\section{Collaboration with Faculty Sponsor}
My primary faculty sponsor for this project is Dr. Brittany Fasy, whose ongoing
research in 3D shape analysis aligns with the methodological aspects of this
project. I plan to meet with Dr. Fasy regularly as her expertise in computational 
geometry and 3D data analysis will be invaluable in guiding the development and 
refinement of algorithms for this project. Additionally, I will be advised by
Dr. Chris Organ, whose work in evolutionary biology and paleontology 
will provide crucial insights into the biological relevance of the morphological 
features I am going to analyze. I am eager to engage in this research project, 
which not only supports the work of my mentors but also aims to make a meaningful 
contribution to the fields of computational topology and evolutionary biology.

\newpage

% Section 6: References
\bibliography{ref.bib} 

\end{document}


